\subsection{Forme normale}

Comme on a pu le voir précédemment, les \lexps{} impliquent un certain nombre de réécritures via les \breds{}. Ainsi, on peut définir la forme normale d'un \lterm{} comme étant le terme réduit sur lequel on ne peut plus appliquer de réductions. Obtenir une telle forme est cruciale puisque l'algorithme d'unification que nous allons tenter d'implémenter travaille sur ces termes.
Nous allons voir ici 2 notions de forme normale pour les \lterms{} : \textit{$\beta$-normal form} et \textit{$\eta$-long normal form}.

\subsubsection{$\beta$-normal form}

\begin{defn}
Soit $a$ un \lterm{} correctement typé et de la forme \textit{$\beta$-normal}, alors $a$ est de la forme :
\[
    \lambda x_1 \dots \lambda x_n . h . e_1 \dots e_p
\]
avec $n, p \geq 0$, $h$ une constante, une variable liée ou une méta-variable, appelée \textit{tête} de $a$ et $e_1 \dots e_p$ sont des \lterms{} $\beta$-normalisés appelés \textit{arguments} de $h$.
\end{defn}

Il existe plusieurs qualifications aux $\beta$ normal forms (que l'on abrègera à partir de maintenant en \textit{bnfs}) que nous allons maintenant définir.

\begin{defn}
Un \lterm{} \textit{bnf} est dit \textit{rigid} si sa tête est une constante ou une variable liée. Sinon, on dit qu'il est \textit{flexible}, sa tête est alors une méta-variable.
\end{defn}

Pour obtenir la $\beta$-normal form, il faut appliquer la règle de réduction $\beta$ jusqu'a ce que l'on ne puisse plus réduire. cela fonctionne dans le \lc{} car il est confluant avec la $\beta$-réduction, ie. il n'existe qu'une seule forme normale.
\noindent
\\
Dans le cadre du \lsc{} des règles de réduction supplémentaires sont nécessaires, les règles réductions $\sigma$ travaillent sur les substitution explicites.
Cependant, avec l'introduction des méta-variables, la règle $\beta$ et les règles $\sigma$ ne suffisent pas pour assurer la confluence du \lsc{}.
L'utilisation de la règle $\eta$ accompagné d'un système de types simples permet de rendre le \lsc{} confluent.
~\\
\\
Les règles de réductions du \lsc{} sont :
\begin{description}
    \item \textbf{Beta} : \[ (\lambda a) b \xrightarrow{} a[b.id] \]
    \item \textbf{App} : \[ (a \, b)[s] \xrightarrow{} (a[s] \, b[s]) \]
    \item \textbf{VarCons} : \[ 1[a.s] \xrightarrow{} a \]
    \item \textbf{Id} : \[ a[id] \xrightarrow{} a \]
    \item \textbf{Abs} : \[ (\lambda a)[s] \xrightarrow{} \lambda ( a [1 . (s \circ \uparrow)]) \]
    \item \textbf{Clos} : \[ (a[s])[t] \xrightarrow{} a[s \circ t] \]
    \item \textbf{IdL} : \[ id \circ s \xrightarrow{} s \]
    \item \textbf{ShiftCons} : \[ \uparrow \circ (a.s)\xrightarrow{} s \]
    \item \textbf{AssEnv} : \[ (s_1 \circ s_2) \circ s_3 \xrightarrow{} s_1 \circ (s_2 \circ s_3) \]
    \item \textbf{MapEnv} : \[ (a.s) \circ t \xrightarrow{} a[t] . (s \circ t) \]
    \item \textbf{IdR} : \[ s \circ id \xrightarrow{} s \]
    \item \textbf{VarShift} : \[1.\uparrow \xrightarrow{} id \]
    \item \textbf{Scons} : \[ 1[s].(\uparrow \circ s) \xrightarrow{} s \]
    \item \textbf{Eta} : \[ \lambda (a \, 1) \xrightarrow{} b \text{ si $a=_\sigma b[\uparrow]$} \]

\end{description}

\paragraph{Implémentation}

L'implémentation de ces règles est :
\begin{lstlisting}
let beta_red_s (t: s_term) =
  match t with
  | S_App (S_Abs (ty, a), b) -> S_Tsub (a, Cons (b, Id))
  | _ -> t

let eta_red_s (t: s_term) =
  match t with
  | S_Abs (ty, S_App(a, S_One)) -> unshift_s a
  | _ -> t

(* All reduction rules for terms *)
let reduce_term_s (t : s_term) =
  match t with
  | S_Tsub (S_App (a,b), s) -> S_App (S_Tsub (a,s), S_Tsub (b,s))
  | S_Tsub (S_One, Cons (a,b)) -> a
  | S_Tsub (a, Id) -> a
  | S_Tsub (S_Abs (ty, a), s) -> S_Abs (ty, S_Tsub (a, Cons (S_One, Comp (s, Shift))))
  | S_Tsub (S_Tsub (a,s),t) -> S_Tsub (a, Comp (s,t))
  | _ -> t

(* All reduction rules for subst *)
let reduce_subst_s (s : s_subst) =
  match s with
  | Comp (Id, s1) -> s1
  | Comp (Shift, Cons (a,s1)) -> s1
  | Comp (Comp (s1, s2), s3) -> Comp (s1, Comp (s2,s3))
  | Comp (Cons (a,s1),t) -> Cons (S_Tsub (a,t), Comp (s1,t))
  | Comp (s1, Id) -> s1
  | Cons (S_One, Shift) -> Id
  | Cons (S_Tsub (S_One, s1), Comp (Shift, s2)) -> if s1 = s2 then s1 else s
  | _ -> s
\end{lstlisting}

\begin{prop}
Les propriétés du \lsc{} (sans types et avec la règle $\eta$) sont :
\begin{enumerate}
    \item Le \lsc{} est localement confluent sur n'importe quel \lsterm{}, ouvert ou fermé.
    \item Le \lsc{} est confluent sur les termes sans variables de substitutions
    \item Le \lsc{} n'est pas confluent sur les termes avec des variables de substitutions.
\end{enumerate}
\end{prop}

\subsubsection{$\eta$-long normal form}
Dans cette partie nous allons définir la notion de $\eta$-long normal form utilisée par Dowek pour tenter de résoudre le problème de l'unification dans \cite{dowek1995higher}.

\begin{defn}
Soit $a$ un \lsterm{} de type $A_1 \xrightarrow{} \dots \xrightarrow{} A_n \xrightarrow{} B$ dans le contexte $\Gamma$ et de la forme $\lambda\sigma$-normal. La $\eta$-long normal form de $a$, notée $a'$, est défini par :
\begin{enumerate}
    \item Si $a = \lambda_Cb'$ alors $a' = \lambda_Cb'$.
    \item Si $a = (\verb?k?b_1\dots b_p)$ alors $a' = \lambda_{A_1} \dots \lambda_{A_n}(k+n\ c_1\dots c_p\  \verb?n'?\dots \verb?1'?)$ avec $c_i$ la $\eta$-long normal form de la forme normale de $b_i[\uparrow^n]$.
    \item Si $a = (X[s]b_1\dots b_p$ alors $a' = \lambda_{A_1}\dots \lambda_{A_n}(X[s']\ c_1\dots c_p\ \verb?n'?\dots \verb?1'?)$ avec $c_i$ la $\eta$-long normal form de la forme normale de $b_i[\uparrow^n]$ et si $s = d_1\dots d_q.\uparrow^k$ alors $s' = e_1\dots e_q.\uparrow^{k+n}$ où $e_i$ est $\eta$-long form de $d_i[\uparrow^n]$.
\end{enumerate}


\begin{defn}
La \textit{long normal form} d'un terme est la \textit{$\eta$-long form} de sa $\beta\eta$-normal form.
\end{defn}

\begin{prop}
Deux termes sont $\beta\eta$-équivalents si et seulement si ils ont la même \textit{long normal form}. 
\end{prop}

\paragraph{Implémentation}

Voici l'implémentation de la $\eta$-long normal forme que nous utilisons pour notre fonction d'unification (TODO METTRE LA REF).

\begin{lstlisting}
let rec eta_long_normal_form (t : s_term) (typ : ty) (m_c : meta_var_str) : s_term =
  match t with
  | S_One | S_Xvar _ | S_Tsub (_,_) -> t
  | S_App (t1,t2) -> 
        let (ty,rest) = get_before_last_type typ in
        let left = (match t1 with
            | S_One -> S_Tsub (S_One,(s_shift_n 2))
            | S_Tsub(S_One,s1) -> S_Tsub(S_One,Comp(Shift,s1))
            | S_Tsub(S_Xvar n,s1) ->
               let meta_type = fst (Map_str.find n m_c) in 
               let s_prime = 
                 apply_fun_subst s1 (fun t -> eta_long_normal_form t meta_type m_c) in S_Tsub(S_Xvar n,s_prime)
            | _ -> failwith "eta_long_normal_form can't happend you should be in normal form") in
        let right = eta_long_normal_form (normalise_lambda_sigma (S_Tsub(t2,(s_shift_n 1)))) rest m_c in 
                     S_Abs(ty,S_App(left,right))
  | S_Abs (ty1,t1) ->
     S_Abs(ty1,eta_long_normal_form t1 typ m_c)
\end{lstlisting}

La fonction est très proche de la définition cependant ont nécessite simplement une
fonction auxiliaire afin de récupérer l'avant dernier type d'un type flèche :
\verb|get_before_last_type| . La seule différence avec la définition est que au lieu de 
considérer des applications n-aires on ne peut considérer qu'une application à la fois.


\end{defn}