\subsection{Typage}


% PRO TIPS : \lc{} => écrit lambda-calcul avec le symbole
Afin de réaliser l'unification, les types sont nécessaires.
TODO NOUS EN REPARLERONS PLUS TARD METTRE LE LIEN
Les règles pour les termes sont les mêmes que celles du \lc{} simplement typé. Voici les règles impliquant des substitutions:
 
\textbf{Clos}:
\[ \frac{\Gamma \vdash s \rhd \Gamma' \quad \quad \Gamma' \vdash a:A}{\Gamma \vdash a[s]:A} \]

\textbf{Id}:
\[ \Gamma \vdash id \rhd \Gamma \]

\textbf{Shift}:
\[ A.\Gamma \vdash \uparrow \rhd \Gamma \]

\textbf{Cons}:
\[ \frac{\Gamma \vdash a:A \quad \quad \Gamma \vdash s \rhd \Gamma'}{\Gamma \vdash a.s \rhd A.\Gamma'}\]

\textbf{Comp}:
\[ \frac{\Gamma \vdash s'' \rhd \Gamma'' \quad \quad \Gamma'' \vdash s'\rhd \Gamma'}{\Gamma \vdash s' \circ s'' \rhd \Gamma'} \]
 
\begin{prop}
Le \lsc{} typé est :
\begin{itemize}
    \item confluent
    \item faiblement normalisable
\end{itemize}
\end{prop}
 
L'algorithme de typage que nous utilisons est celui du \lc{} simplement typé avec inférence de type.


\paragraph{Implémentation}

Focalisons nous simplement sur l'implémentation des règles concernant les substitutions :

\begin{lstlisting}
and type_check_cont c t_sub m_c =
  match t_sub with
  | Id           -> c
  | Shift        -> (match c with
      | []     -> raise No_inference
      | hd::tl -> tl)
  | Cons(t, s)   -> let c_s = type_check_cont c s m_c in
    let ty_t = type_check_inf c t m_c in
    ty_t::c_s
  | Comp(s1, s2) -> let c_s2 = type_check_cont c s2 m_c in
    type_check_cont c_s2 s1 m_c 
\end{lstlisting}



