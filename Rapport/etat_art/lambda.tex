\subsection{$\lambda$-calcul}

Nous allons tout d'abord commencer par définir plusieurs notions de \lc{} qui est l'élément de base sur lequel nous travaillons. 

\subsubsection{$\lambda$-calcul nommé}

Soit $V$ un ensemble de variables ($V={x,y,\dots}$). Les termes du $\lambda$-calcul sont définis par :
\[ a ::= x \,|\, (a \, a) \,| \,\lambda x. a\]

\paragraph{}
Le terme $\lambda x.a$ représente la fonction qui a $x$ associe $a$ (le corps de la fonction), et le terme $(M \, N)$ représente le résultat de la fonction $M$ appliquée à $N$. Si on applique la fonction $\lambda x.a$ à un terme $b$, alors on obtient la valeur de $a$ pour laquelle toutes les occurrences de $x$ ont été remplacées par $b$. Cette opération est appelée $\beta$-réduction. Avant d’effectuer une $\beta$-réduction, les variables liées qui apparaissent à la fois dans $a$ et dans $b$ doivent se voir subir une $\alpha$-conversion: on renomme ces variables pour $a$ ou $b$ de sorte qu'aucune variable ne soit capturée par mégarde. Une autre règle, la $\eta$-réduction, est aussi ici introduite.

\begin{defn}
Soit $a$ un terme et $FV(a)$ l’ensemble des variables libres de $a$. Pour une évaluation $\theta$ liant les variables $x_1,\dots,x_n$ aux variables $a_1,\dots,a_n$, la substitution qui étend $\theta$ notée $\bar{\theta}={x_1/a_1,\dots,x_n/a_n}$ est définie par:
\begin{align*}
&\bar{\theta}x = \theta x \\
&\bar{\theta}(a \, b) = (\bar{\theta}a \, \bar{\theta}b) \\
& \bar{\theta}(\lambda y.a) = \lambda z (\bar{\theta}\{y/z\}a) 
\end{align*}
où $z$ est une nouvelle variable telle que $\theta z = z$, $z$ n'apparaît pas dans $a$, et $\forall{x} \in FV(a) \quad z \notin FV(\theta x)$.
\end{defn}

\begin{defn}
La $\beta$-réduction est définie par la règle:
\[(\lambda x.a)b \xrightarrow{} \{x/b\}a\]
La $\eta$-réduction est définie par la règle:
\[\lambda x.(a \, x) \xrightarrow{} a \quad \text{si } x \notin FV(a) \]
\end{defn}

\subsubsection{Les variables d’unification}

Dans un $\lambda$-terme, on peut distinguer trois types de variables:
\begin{itemize}
    \item Les variables libres
    \item Les variables liées
    \item Les vraies variables d’unification qui définissent le problème d’unification.
\end{itemize}

\[ v::= x \, | \, X \]

Ces dernières seront appelées les \textit{métavariables}. On notera $V$ l'ensemble des variables libres et liées, et $H$ l'ensemble des métavariables.

L'introduction de ces dernières nous permet alors de définir de manière plus précise les termes du $\lambda$-calcul:

\begin{defn}
L'ensemble $\Lambda(V,H)$ des $\lambda$-termes ouverts est défini par:
\[ a ::= x \,|\, X \,|\, (a \, a) \,| \,\lambda x. a\]
où $x \in V$ et $X \in H$.\\
\end{defn}

Nous avons donc maintenant deux notions de substitution:
\begin{itemize}
    \item la première fonctionne sur $V$: elle est utilisée pour la $\beta$-réduction.
    \item la seconde fonctionne sur $H$: elle est utilisée pour l'unification.
\end{itemize}

Le même problème que celui cité précédemment apparaît encore: lors d'une substitution, des variables libres peuvent être capturées par des \textit{binders} certaines d'entre elles peuvent être capturées, comme l'illustre l'exemple suivant:
\[\{ X \mapsto x \}(\lambda x . X) = \lambda x . x\]

Nous devons donc de nouveau définir la notion de substitution avec renommage des variables liées:

\begin{defn}
Soit $\theta$ une évaluation (une fonction de $H$ dans $\Lambda(V,H)$. La substitution $\bar{\theta}$ est l'extension de cette évaluation telle que:
\begin{enumerate}
    \item $\bar{\theta} (X) = \theta (X)$
    \item $\bar{\theta} (x) = x$ si $x \in V$
    \item $\bar{\theta}(a \, b) = (\bar{\theta}(a) \, \bar{\theta}(b))$
    \item $\bar{\theta}(\lambda y . a) = \lambda z . \bar{\theta} (\{ y / z \} a)$ où $z$ est une nouvelle variable.
\end{enumerate}
$\bar{\theta}$ sera maintenant notée $\theta$.
\end{defn}

\begin{prop}
Les substitutions et les réductions commutent.
\end{prop}

\subsubsection{$\lambda$-calcul avec la notation de de Bruijn}

Dans le $\lambda$-calcul nommé vu précédemment, on nomme toujours les variables liées alors que ces noms ne sont en rien utiles à notre problème.Cette gestion du renommage des variables, qui peut parfois être très complexe, peut être évitée en utilisant une autre formulation du $\lambda$-calcul: la notation de de Bruijn. On en présente ici le principe (en ajoutant directement les métavariables).

\begin{defn}
L’ensemble $\Lambda_{DB}(H)$ des $\lambda$-termes dans la notation de de Bruijn est défini par:

\[ a::= n \, |\, X \,| \,\lambda \, a \,|\, (a \, a) \]

où $n$ est un entier supérieur ou égal à $1$ et $X \in H$.
\end{defn}

On a donc remplacé ici les variables liées par des indices, mais on conserve les noms pour les métavariables. De plus, la notation de de Bruijn introduit aussi la notion de référentiel: les $V$-variables (i.e. les variables liées, mais pas les métavariables) sont ordonnées dans une liste $(x_0 \dots x_n)$ qui définit le référentiel $R$.

Notre implémentation des lambda-sigma termes en de Brujin est similaire à la définition ci-dessus : 

\paragraph{Implémentation}

\begin{lstlisting}
(* lambda terms *)
type term =
  | Var  of int
  | XVar of name
  | Abs  of ty * term
  | App  of term * term
\end{lstlisting}

\begin{defn}
Soit $R$ un référentiel. Soit $a \in \Lambda(V,H)$ tel que toutes les $V$-variables libres de $a$ sont déclarées dans $R$. La translation de de Bruijn de $a$, notée $tr(a,R)$ est définie par:
\begin{enumerate}
    \item $tr(x,R) = j$, où $j$ est l’indice de la première occurence de x dans $R$
    \item $tr(X,R) = X$
    \item $tr((a \, b),R) = (tr(a,R) \, tr(b,R))$
    \item $tr(\lambda x . a, R) = \lambda (tr(a, x . R))$
\end{enumerate}
\end{defn}

Ainsi, pendant une translation de de Bruijn, le référentiel est incrémenté lorsque l’on rencontre un $\lambda$. Par exemple:

\[ \lambda x \, \lambda y . (x ( \lambda z . (z \, x)) y) \] 
s’écrit maintenant:

\[ \lambda ( \lambda (2 (\lambda 1 \, 3) 1 )) \]

\begin{defn}
La $\lambda-longueur$ d’une occurrence $u$ dans un terme $a$ est le nombre d'abstractions précédant $u$. Elle est notée $|u|$.
\end{defn}

On doit maintenant définir la substitution qui correspond à la substitution des éléments de $V$.

\begin{defn}
Soit $a \in \Lambda_{DB}(H)$. Le terme $a^{+}$, appelé \textit{lift} de $a$, est défini par $a^{+} = lft(a,0)$ où $lft(a,i)$ est défini par:
\begin{enumerate}
    \item $lft((a_1 \, a_2),i) = (lft(a_1,i) \, lft(a_2,i))$
    \item $lft(\lambda a, i) = \lambda (lft(a,i+1))$
    \item $lft(X,i) = X$
    \item $lft(m,i) = m + 1$ si $m > i$, $m$ sinon.
\end{enumerate}
\end{defn}

On définit maintenant l’analogue de la $V$-substitution de $\Lambda(V,H)$.

\begin{defn}
La substitution par $b$ à la $\lambda$-longueur $(n-1)$ dans $a$, notée $\{n/b\}a$, est définie par:
\begin{align*}
\{n/b\}(a_1 \, a_2) &= (\{n/b\} a_1 \, \{n/b\} a_2) \\
\{n/b\} X &= X \\
\{n/b\} \lambda a &= \lambda (\{n+1/b^{+}\} a) \\
\{n/b\} m &= m-1 \quad \text{si $m > n$ ($m \in FV(a)$)} \\
&\quad \quad b \quad \text{si $m=n$ ($m$ lié par le $\lambda$ du $\beta$-redex)} \\
&\quad \quad m \quad \text{si $m < n$ ($m \in BV(a)$)}
\end{align*}
\end{defn}
%lift incrémente les variables libres d'un terme de un
%lambda x (lambda y x)
%x -> z
%z->3
%lambda 1 (lambda 1 2)
%[0->3](lambda 1 (lambda 1 2))
%3 (lambda 1(lambda  (1 5))
%z->4%

% subst remplace une variable libre(+1 n fois pour n binders), par un terme(+1 free vars n fois pour n binders)
% subst décremente toute les variables libres d'un terme pour la béta-réduction
%(lambda x (lambda y t x)) z
%(lambda y z)
%( lambda 1 (lambda 1 3 2) ) 3
%(lambda 1 2 5)


\paragraph{Implémentation}

Voici notre implémententation du lift et de la substitution :
\begin{lstlisting}
let rec lift (t : term) (i : int) =
  match t with
  | Var(n) -> if n > i then Var(n+1) else t
  | XVar(_) -> t
  | Abs(ty, t) -> Abs(ty, lift t (i+1))
  | App(t1, t2) -> App((lift t1 i), (lift t2 i))

(* lift (up arrow) operation : increments free de brujin indices *)
let lift_plus (t : term) = lift t 0

(* substitution for beta reduction *)
let rec subst (a: term) (b: term) (n: int) =
  match a with
  | Var(m) -> if m > n then Var(n-1)
    else (if n = m then b else Var(m))
  | XVar(_) -> a
  | Abs(ty, t1) -> Abs(ty, subst t1 (lift_plus b) (n+1))
  | App(t1, t2) -> App((subst t1 b n), (subst t2 b n))
\end{lstlisting}

\begin{defn}
La $\beta$-réduction, dans ce contexte, est définie naturellement par:

\[ ((\lambda a)b) \xrightarrow{}{\{1/b\}} a \]

où $\{1/b\}$ est la substitution de l’indice $1$ par le terme $b$.
\end{defn}

Par exemple, le terme $\lambda x . (( \lambda y . (x \, y)) x)$ s’écrit $\lambda (( \lambda (2 \, 1))1)$ dans la notation de de Bruijn. Il se réduit en $\lambda x . (x \, x)$, ou, dans la notation de de Bruijn, en $\lambda (\{1/1\}(2 \, 1)) = \lambda (1 \, 1)$.

\begin{defn}
La $\eta$-réduction dans $\Lambda_{DB}(H)$ est définie par la règle:

\[ \lambda (a \, 1) \xrightarrow{}{b} \quad \text{si $b \in \Lambda_{DB}(H)$ est tel que $a = b^{+}$.} \]

\end{defn}

\begin{prop}
Pour un terme $a$ de $\Lambda_{DB}(H)$, il existe un terme $b$ tel que $a=b^{+}$ si et seulement si, pour toute occurrence $u$ d’indice $p$ dans $a$, $p \neq |u| + 1$.
\end{prop}

Enfin, nous définissons la $H$-substitution. $\theta^{+}$ est défini par $\theta^{+} = \{X_1/a_1^{+}, \dots, X_n/a_n^{+}\}$ quand $\theta = \{X_1/a_1, \dots , X_n/a_n\}$.

\begin{defn}
Soit $\theta$ une évaluation de $H$ dans $\Lambda_{DB}(H)$. La substitution $\bar{\theta}$ associée est définie par les règles:
\begin{enumerate}
    \item $\bar{\theta} (X) = \theta (X)$
    \item $\bar{\theta} (n) = n$
    \item $\bar{\theta} (a_1,a_2) = (\bar{\theta} (a_1) \bar{\theta} (a_2))$
    \item $\bar{\theta} (\lambda a) = \lambda (\bar{\theta}^{+} (a))$
\end{enumerate}
\end{defn}

La relation entre la substitution dans le $\lambda$-calcul simple et la substitution dans le $\lambda$-calcul avec les indices de de Bruijn s’exprime selon la proposition suivante:

\begin{prop}
Soit $a \in \Lambda(V,H)$ et $\theta = \{ X_1/a_1, \dots, X_n / a_n \}$ un terme et une substitution d’un $\lambda$-calcul simple. Soit $\theta^{r} = \{ X_1 / tr(a_1,R), \dots, X_n / tr(a_n,R) \}$. Alors $tr(\theta (a),R) = \theta^{r}(tr(a,R))$.
\end{prop}

\subsubsection{Le $\lambda \sigma$-calcul}

Le $\lambda \sigma$-calcul implémente le $\lambda$-calcul écrit avec la notation de de Bruijn, ainsi que les $\beta$ et $\eta$ réductions. Il est différent de part la manière dont ils traitent les substitutions. Essayons de comprendre l’origine du $\lambda \sigma$-calcul. \\

Soit $F$ une fonction de $A$ dans $A$. Par exemple: $F(0)=0$, $F(S(0))=S(S(0))$, $F(S(S(0)))=S(S(S(S(0))))$, etc. Une autre manière de définir cet opérateur va consister à introduire une nouvelle fonction intermédiaire $f$ permettant de placer l’opération $F$ en interne. Nous allons aussi introduire des règles permettant de réécrire les expressions contenant $f$ en d’autres expressions ne contenant pas forcément $f$. Par exemple:
\begin{align*}
f(0) &\xrightarrow{}{0} \\
f(S(X)) &\xrightarrow{}{S(S(f(x)))}
\end{align*}
Dans le $\lambda$-calcul simple, les applications sont déjà placées en interne, mais les substitutions et le \textit{lifting} se traitent encore de façon externe. Le $\lambda \sigma$-calcul, lui, place de façon interne les substitutions et le \textit{lifting}.\\

\paragraph{Notations utiles.} Les listes sont représentées par un opérateur \textit{cons} (noté « . ») et un opérateur pour la liste vide (noté $id$). La substitution $a_1.a_2. \dots a_n . id$ remplace 1 par $a_1$, $\dots$, $n$ par $a_n$, et décrémente par $n$ tous les autres indices (libres) dans le terme. L’opérateur qui internalise le \textit{lifting} est noté $\uparrow$, et peut être vu comme une substitution infinie $2.3.4.\dots$. On introduit également un opérateur de composition $\circ$. Ainsi, l'indice $n+1$ s'écrit:

\[1 [\uparrow \circ \dots \circ \uparrow] = 1 [ \uparrow^n] \]

Les termes du $\lambda \sigma$-calcul sont alors construits de la manière suivante:

\begin{defn}
Soit $X$ un ensemble de métavariables de termes et $Y$ un ensemble de métabariables de substitution. L'ensemble $I_{\lambda \sigma}(H,Y)$ des termes et des substitutions explicites est défini par:
\begin{align*}
a &::= 1 \, | \, x \, | \, (a \, a) \,| \, \lambda a \, | \, a[s] \\
s &::= y \, | \, id \, | \, \uparrow \,| \, a.s \, | \, s \circ s
\end{align*}
où $x \in X$ et $y \in Y$.
\end{defn}

\begin{prop}
Chaque $\lambda \sigma$-terme dans sa forme normale pour le $\lambda \sigma$ est d'une des formes suivantes:
\begin{enumerate}
    \item $\lambda \, a$ où $a$ est une forme normale
    \item $(a \, b_1 \dots b_p)$ où $a$ et les $b_i$ sont des formes normales, et $a$ est soit $1$, $1[\uparrow^n]$, $X$, ou $X[s]$ où $s$ est un terme de substitution dans sa forme normale différent de $id$
    \item $a_1 \dots a_p \cdot \uparrow^n$ où $a_1, \dots ,a_p$ sont dans leurs formes normales et $a_p \neq n$
\end{enumerate}
\end{prop}

\begin{prop}
Le $H$-grafting et la $\lambda \sigma$-réduction commutent.
\end{prop}

\paragraph{Implémentation}

Pour notre implémentation, nous n'avons pas eu besoin de métavariables de substitution. L'implémentation des \lsterm{} est :
\begin{lstlisting}
(* lambda sigma substitutions and terms *)
type s_subst =
  | Id
  | Shift
  | Cons of s_term * s_subst
  | Comp of s_subst * s_subst
and s_term =
  | S_One
  | S_Xvar of name
  | S_App of s_term * s_term
  | S_Abs of ty * s_term
  | S_Tsub of s_term * s_subst
\end{lstlisting}