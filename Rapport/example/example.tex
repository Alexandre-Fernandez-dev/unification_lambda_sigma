\section{Unification \textit{higher order} par l'exemple}

Nous allons, dans cette partie, résoudre un problème simple d'unification pour illustrer le fonctionnement de l'algorithme.

Soit le problème suivant, encodé en de Bruijn, que l'on va essayer de résoudre :
\[
\lambda_Ay.(Xa) =^{?}_{\beta\eta} \lambda_Ay.a
\]
avec
\[
a : X
\]
\[
X : A\xrightarrow{}A
\]
Cette équation est encodé en de Bruijn en utilisant le contexte suivant :
\[
\Gamma = A.nil
\]
On obtient alors :
\[
\lambda(X \verb?2?) =^{?}_{\beta\eta} \lambda\verb?2?
\]
On obtient ensuite après l'étape de \textbf{\textit{precooking}} :
\[
\lambda(X[\uparrow]\verb?2?) \eqls \lambda\verb?2?
\]
On applique la règle \textbf{Exp-$\lambda$} :
\[
\exists Y((X[\uparrow]\verb?2?) \eqls \verb?2?) \land (X \eqls \lambda Y)
\]
avec $\Gamma _Y = A.\Gamma$ et $T_Y = A$. En appliquant la règle \textbf{Replace} on a :
\[
\exists Y(((\lambda Y)[\uparrow]\verb?2?) \eqls \verb?2?) \land (X \eqls \lambda Y)
\]
Avec \textbf{Normalize} on a :
\[
\exists Y(Y[\verb?2?.\uparrow] \eqls \verb?2? \land (X \eqls \lambda Y)
\]
En appliquant la règle \textbf{Exp-App} on a :
\[
\exists Y(Y[\verb?2?.\uparrow] \eqls \verb?2?) \land (X \eqls \lambda Y) \land (Y = \verb?1?)
\]
\[
\lor
\]
\[
\exists Y(Y[\verb?2?.\uparrow] \eqls \verb?2?) \land (X \eqls \lambda Y) \land (Y = \verb?2?)
\]
On réduit ensuite avec la règle \textbf{Replace}
\[
(\verb?1?[\verb?2?.\uparrow] \eqls  \verb?2?) \land (X \eqls \lambda \verb?1?) \lor (\verb?2?[\verb?2?.\uparrow] \eqls \verb?2?) \land (X \eqls \lambda\verb?2?)
\]
Enfin, en normalisant avec la règle \textbf{Normalize}, on a :
\[
(X \eqls \lambda\verb?1?) \lor (X \eqls \lambda\verb?2?)
\]
On a donc 2 solutions possibles au problème posé initialement.