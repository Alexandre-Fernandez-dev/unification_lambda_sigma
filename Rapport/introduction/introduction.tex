\section{Introduction}

Dans le cadre de l'enseignement de Groupe de Recherche en Algorithmique, il va s'agir dans ce projet, qui s'inscrit dans le développement de l'assistant de preuve LaTTe\footnote{\url{https://github.com/latte-central/latte-checker}} par Frédéric Peschanski, d'implémenter un algorithme d'unification d'ordre supérieur.

L'unification est une opération fondamentale dans les algorithmes de déduction et consiste en la recherche d'une substitution entre un ou plusieurs termes afin de les rendre égaux. L'unification est en quelque sorte une équation à résoudre, on dit alors que les termes sont unifiables si une telle substitution est possible.

L'unification présente plusieurs aspects, notamment l'unification du premier ordre qui est un problème que l'on sait déjà très bien résoudre. Ici on va s'intéresser à l'unification d'ordre supérieur (ou \textit{higher order unification}) qui est un problème semi-décidable puisque l'algorithme se termine pour des entrées qui ont une solution et peut boucler pour des entrées sans solution.